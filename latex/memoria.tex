\documentclass[12pt]{article}
\usepackage[utf8]{inputenc}
\usepackage{amsmath}
\usepackage{graphicx}
\usepackage{hyperref}
\usepackage[spanish]{babel}

\title{\Huge Visión Artificial}
\author{\Large Juan Diego Gallego Nicolás\\ \href{mailto:jdiego.gallego@um.es}{jdiego.gallego@um.es}}
\date{\Large 23/03/2025}

\begin{document}

\maketitle
\thispagestyle{empty}

% Portada
\begin{center}
    \vspace{2cm}
    \textbf{Entrega Parcial}
\end{center}

\newpage

% Índice
\tableofcontents
\newpage

% Sección 1: Introducción
\section{Introducción}
Este es un párrafo de ejemplo para la introducción. Aquí se puede describir el contexto general del trabajo, los objetivos que se buscan alcanzar y una breve descripción de los temas que se tratarán.

\newpage

% Sección 2: Calibración
\section{Calibración}
En esta sección se puede hablar sobre el proceso de calibración de las cámaras o sensores utilizados en el trabajo. La calibración es fundamental para obtener resultados precisos en la visión artificial. Se debe incluir la explicación de los métodos de calibración empleados y los resultados obtenidos.

\newpage

% Sección 3: Filtros
\section{Filtros}
En esta sección se describen los diferentes filtros utilizados en el proceso de procesamiento de imágenes. Los filtros son herramientas esenciales para mejorar la calidad de las imágenes antes de aplicar técnicas de visión artificial más avanzadas.

\newpage

% Sección 4: Clasificador
\section{Clasificador}
Aquí se hablará sobre los clasificadores utilizados para la tarea específica. Se puede incluir la descripción de los algoritmos empleados, su rendimiento y los datos con los que se entrenaron.

\end{document}